\chapter{Internet of Things}

% Internet of things paradigm
% smart environment
% smart cities

% As identified by Atzori et. al. [7], Internet of Things can be realized in three paradigms – internet-oriented (middleware), things oriented (sensors) and semantic-oriented (knowledge).

Claiming to have coined the phrase "Internet of Things" back at a presentation for Procter \& Gamble in 1999, Kevin Ashton \cite{ashton_iot} introduced the world to a phrase that almost 20 years later has gained a lot of momentum and attention. Even though the Internet of Things Global Standards Initiative (IoT-GSI) in 2012 presented a recommended definition of IoT \cite{itu-t:Y.2060}, research literature presents a wide array  of different definitions\cite{Gubbi:2013:ITV:2489313.2489456} \improvement{citations}. The wide range of definitions may be an indicator of the strong interest that IoT has received and the lively debates around the topic. Another reason is the syntactic structure of the phrase itself. Being built up of the terms 'Internet' and 'Things' leads to definitions that either has an "Internet oriented" or a "Things oriented" focus, which in turn creates substantial differences in the definitions \cite{Atzori:2010:ITS:1862461.1862541}. Together the  terms produces the semantical meaning of "a world-wide network of interconnected objects uniquely addressable, based on standard communication protocols"\cite{eposs:iot-defination-2008}. 

Atzori et. al \cite{Atzori:2010:ITS:1862461.1862541} further introduces the "Semantic orientation", which relates to uniquely addressing an


% Growth in IoT: According to a Juniper Research study released last month, IoT platform-based devices in the retail environment, including RFID and Bluetooth Low Energy (BLE) beacons, will number 12.5 billion in the next four years. The figure is rising 350 percent, the study finds, from the 2.7 billion connected devices last year. These devices include RFID tags on products, Bluetooth beacons installed in stores, and digital signage or electronic shelf labels. The details are available in the study, titled "IoT in Retail: Strategies for Customer Experience, Engagement & Optimisation 2017 – 2021." http://www.rfidjournal.com/articles/view?15929

\section{Pervasive/Ubiquitous Computing}
\section{Cloud}

% Pros and con's of the Cloud and the combination of 'Things' and the Cloud. 

\subsection{Cloud services}

% Azure
% AWS
% Google?

\subsection{The Fog}

\cite{cisco:fog}
% Moving computational power out to the edge, providing quicker and more reliable service. 
% Reduces response times and network traffic, reduces data processing in the cloud as data can be aggregated in the fog before it's sent to the cloud. 
% Similar to IoT in that it monitors, analyzes and acts upon real-time data from network-connected things.